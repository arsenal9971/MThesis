\chapter*{Introduction}
\addcontentsline{toc}{chapter}{Introduction}

This thesis is about point processes applied to telecommunication models. It is divided in three different main parts. The first part (Chapter 1) is a brief introduction to the theory and main results of point processes that will be used in the rest of the work. The second part (Chapters 2 and 3) is about the study of the traffic flow densities and its mainly based on \cite{WIAS}. Finally in the third part (Chapter 4) based in \cite{Iye} we expose the concept of information velocity.

In chapter one first we define what is a point process, we explain some details about its construction and we proof one of the main results related to them, which allow us to compute expected value of a random sum evaluated on the points of a point process: the Campbell's theorem. We also define what is a stationary point process an the related concept of nearest neighbor density.  Also we state an ergodic theorem for stationary point processes. Second we focus on the most classical example of point processes: the Poisson point process, which independence properties makes simply to characterize and study, we also mention a more complete version of Campbell's theorem and even a generalization of it: the Slyvniak-Mecke's theorem. We also compute the nearest neighbor density of a Poisson point process. Due to its characteristics the Poisson point process is quite often used in telecommunication models. Third we give a brief introduction to the concept of Hausdorff measure and its relation with the coarea formula.

Second chapter is about directed navigations and traffic flows. A directed navigation is a function $\mathcal{A}$ that assigns to each point of a point process a successor in the direction $e_1$. A traffic flow is a function that designates some weight to each node according to the location of the other nodes that are on its tail. Under certain hypothesis for $\mathcal{A}$ a limiting result about the average of the traffic flow on an orthogonal window to $e_1$ which collapses to a point is proved in $\cite{WIAS}$. Using similar techniques we prove an analogous result when the window not collapse. Finally also inspired in $\cite{WIAS}$ we provide an example of directed navigation which satisfies the hypothesis of our result: the spanning three navigation. 

Third chapter is about radial navigations and traffic flows. A radial navigation is a function $\mathcal{A}$ that assigns to each point of a point process a successor that is approaching to the origin. As in second chapter we proof a limiting result about the average traffic of nodes that pass through a spherical environment. In $\cite{WIAS}$  is analyzed the case when the environment collapse to a point, here we proof an analogous result when the environment does not collapse.

Finally the fourth chapter is centered in the concept of Signal Interference Noise Ratio (SINR). If we pick two points $x$ and $y$ of a point process $X$ the $SINR_{x,y}(t)$ measures how good or bad is the quality of communication between them at time $t$. If the SINR is bigger that a threshold then the message from $x$ to $y$ is sent if not we try again at time $t+1$. Using a conic strategy which assigns to $x$ to point to which the message will be send we proof two results. The first is a generalization from a result provided in $\cite{Iye}$ to non stationary Poisson point process and it states that the expected value of the time that takes to a message from being send from $x$ to $y$ is finite. The second result was taken from $\cite{Iye}$ and it states that if we continue sending the message between points, the asymptotic limit of the velocity is positive.

For reading this work is not necessary more than the basic mathematics undergraduate knowledge in probability theory. In the first chapter we do not provide proofs of most of the results but this is not relevant for the correct understanding of the following pages.